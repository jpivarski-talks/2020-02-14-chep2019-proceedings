\documentclass{webofc}
\usepackage[varg]{txfonts}

% 6 pages excluding references!

\begin{document}
\title{Awkward Arrays in Python, C++, and Numba}

\author{\firstname{Jim} \lastname{Pivarski}\inst{1,3}\fnsep\thanks{\email{Mail address for first author}} \and
        \firstname{Peter} \lastname{Elmer}\inst{2}\fnsep\thanks{\email{Mail address for second author if necessary}} \and
        \firstname{David} \lastname{Lange}\inst{3}\fnsep\thanks{\email{Mail address for last author if necessary}}
}

\institute{Insert the first address here 
\and
           the second here 
\and
           Last address
          }

\abstract{%
  Insert your english abstract here.
}

\maketitle

\section{Introduction}
\label{intro}
Your text comes here. Separate text sections with

\section{Section title}
\label{sec-1}
For bibliography use \cite{RefJ}

\subsection{Subsection title}
\label{sec-2}
Don't forget to give each section, subsection, subsubsection, and
paragraph a unique label (see Sect.~\ref{sec-1}).

For tables use syntax in table~\ref{tab-1}.
\begin{table}
\centering
\caption{Please write your table caption here}
\label{tab-1}

\begin{tabular}{lll}
\hline
first & second & third  \\\hline
number & number & number \\
number & number & number \\\hline
\end{tabular}

\vspace*{5cm}
\end{table}

% BibTeX or Biber users please use (the style is already called in the class, ensure that the "woc.bst" style is in your local directory)
% \bibliography{name or your bibliography database}
%
% Non-BibTeX users please use
%
\begin{thebibliography}{}
%
% and use \bibitem to create references.
%
\bibitem{RefJ}
% Format for Journal Reference
Journal Author, Journal \textbf{Volume}, page numbers (year)
% Format for books
\bibitem{RefB}
Book Author, \textit{Book title} (Publisher, place, year) page numbers
% etc
\end{thebibliography}

\end{document}
