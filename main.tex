\documentclass{webofc}
\usepackage[varg]{txfonts}
\usepackage{graphicx}
\usepackage{minted}

% 6 pages excluding references!

\begin{document}
\title{Awkward Arrays in Python, C++, and Numba}

\author{%
\firstname{Jim} \lastname{Pivarski}\inst{1}\fnsep\thanks{\email{pivarski@princeton.edu}} \and
\firstname{Peter} \lastname{Elmer}\inst{1}\fnsep\thanks{\email{Peter.Elmer@cern.ch}} \and
\firstname{David} \lastname{Lange}\inst{1}\fnsep\thanks{\email{David.Lange@cern.ch}}}

\institute{Princeton University}

\abstract{%
  Insert your english abstract here.
}

\maketitle

\section{Introduction}

\section{Architecture}

\subsection{High-level Python layer}

\subsection{C++ layer}

\subsection{Numba layer}

\subsection{Kernels layer}

\section{Record-oriented $\to$ columnar}

\subsection{Deeply nested data from ROOT}

\section{Status}

\section{Conclusion}




%% \section{Introduction}
%% \label{intro}
%% Your text comes here. Separate text sections with

%% \section{Section title}
%% \label{sec-1}
%% For bibliography use \cite{RefJ}

%% % BibTeX or Biber users please use (the style is already called in the class, ensure that the "woc.bst" style is in your local directory)
%% % \bibliography{name or your bibliography database}
%% %
%% % Non-BibTeX users please use
%% \begin{thebibliography}{}
%% % and use \bibitem to create references.
%% \bibitem{RefJ}
%% Journal Author, Journal \textbf{Volume}, page numbers (year)
%% \end{thebibliography}

\end{document}
